\documentclass{beamer}
\mode<presentation>


\usepackage[brazil]{babel}
\usepackage[utf8]{inputenc}
 


\usepackage{amsfonts}
\usepackage{amssymb}
\usepackage{amsmath}
\usepackage{algorithm}
\usepackage{algpseudocode}



\usepackage{ae}
\usepackage{graphicx,color}
\usepackage[all]{xy}
\usepackage{empheq}
\usepackage{fancybox}
\usepackage{textcomp}
\usepackage[all]{xy}
\usepackage{textpos}
\usepackage{multicol}
\usepackage{cancel}
\usepackage{listings}
\usepackage{xcolor}
\usepackage{enumerate}
\usepackage{minted}

\usepackage[style=verbose]{biblatex}
\addbibresource{bibtex.bib}

\usepackage{tikz}
\usetikzlibrary{positioning}
\usetikzlibrary{shapes.geometric, arrows}
\usepackage{tkz-graph}
\GraphInit[vstyle = Shade]
\tikzset{
  LabelStyle/.style = { rectangle, rounded corners, draw,
                        minimum width = 2em, fill = yellow!50,
                        text = red, font = \bfseries },
  VertexStyle/.append style = { inner sep=5pt,
                                font = \Large\bfseries},
  EdgeStyle/.append style = {->, bend left} }
\thispagestyle{empty}

\tikzstyle{startstop} = [rectangle, rounded corners, minimum width=3cm, minimum 
\tikzstyle{io} = [trapezium, trapezium left angle=70, trapezium right angle=110, minimum width=3cm, minimum height=1cm, text centered, draw=black, fill=blue!30]
\tikzstyle{process} = [rectangle, minimum width=3cm, minimum height=1cm, text centered, draw=black, fill=orange!30]
\tikzstyle{decision} = [diamond, minimum width=3cm, minimum height=1cm, text centered, draw=black, fill=green!30]
\tikzstyle{arrow} = [thick,->,>=stealth]

\newcommand{\floor}[1]{$\lfloor$ #1 $\rfloor$}

\newcommand\Fontvi{\fontsize{9}{7.2}\selectfont}



\usetheme{Boadilla}

\newcommand{\PC}[1]{\ensuremath{\left(#1\right)}}


\newcommand*{\colorboxed}{}
\def\colorboxed#1#{%
  \colorboxedAux{#1}%
}
\newcommand*{\colorboxedAux}[3]{%
  % #1: optional argument for color model
  % #2: color specification
  % #3: formula
  \begingroup
    \colorlet{cb@saved}{.}%
    \color#1{#2}%
    \boxed{%
      \color{cb@saved}%
      #3%
    }%
  \endgroup
}



\title {Pensando Computacionalmente}

\author[Wladimir Araújo Tavares]{ Wladimir Araújo Tavares$^{1}$  }

\institute[UFC]{$^{1}$Universidade Federal do Ceará - Campus de Quixadá\\}
\date{}
\AtBeginSection[]
{
  \begin{frame}<beamer>{}
    \small
    \tableofcontents[currentsection,currentsubsection]
  \end{frame}
}
\begin{document}

\begin{frame}
	\titlepage
\end{frame}

%%%%%%%%%%%%%%%%%%%%%%%%%%%%%%%%%%%%%%%%%%%%%%%%%%%%%%%%%%%%%%%%%%%%



\begin{frame}{Grafo}

\begin{itemize}
\item \textbf{Objetivos:} Desenvolver o pensamento computacional.

\item \textbf{Público-alvo:}  Alunos a partir do primeiro ano do Ensino Médio.

\item \textbf{Conteúdo:} Grafo

\item \textbf{Tempo:} 50 minutos

\item \textbf{Recursos:} Papel, Caneta.

\end{itemize}
    
\end{frame}


\begin{frame}{Passo 1 - Apresentação da Atividade}

\begin{itemize}
   
\item <1-> Um grafo G é uma estrutura matemática definida por dois conjuntos: um conjunto de vértices denotado por $V$ e um conjunto de arestas denotado por $E$. Geralmente, representamos um grafo como G=(V,E).


\item Os grafos podem ser utilizados para representar uma relação entre os objetos. 

\item Neste caso, os vértices representam os objetos e as arestas representam uma relação entre os objetos. 
\end{itemize}

\end{frame}


\begin{frame}{Passo 1 - Apresentação da Atividade}

Considere a seguinte situação:

\begin{itemize}
   
\item <1-> A distribuição de energia para as diversas regiões do país exige um investimento muito grande em linhas de transmissão e estações transformadoras. 

\item <2->Uma linha de transmissão interliga duas estações transformadoras. Uma estação transformadora pode estar interligada a uma ou mais outras estações transformadoras, mas devido ao alto custo não pode haver mais de uma linha de transmissão interligando duas estações.

\item <3->As estações transformadoras são interconectadas de forma a garantir que a energia possa ser distribuída entre qualquer par de estações. Uma rota de energia entre duas estações $e1$ e $ek$ é definida como uma sequência $(e_1 , l_1 , e_2 , l_2 , \ldots , e_{k-1} , l_{k-1} , e_k )$ onde cada ei é uma estação transformadora e cada $l_i$ é uma linha de transmissão que conecta $e_i$ e $e_{i+1}$.




\end{itemize}



\end{frame}


\begin{frame}{Passo 1 - Apresentação da Atividade}

\begin{itemize}
   
\item <1-> Os engenheiros de manutenção do sistema de transmissão de energia consideram que o sistema está em estado normal se há pelo menos uma rota entre qualquer par de estações, e em estado de falha caso contrário.

\item <2-> Um grande tornado passou pelo país danificando algumas das linhas de transmissão, e os engenheiros de manutenção do sistema de transmissão de energia necessitam de sua ajuda.


\end{itemize}


\end{frame}


\begin{frame}{Passo 1 - Apresentação da Atividade}


\begin{itemize}
   
\item <1-> No exemplo acima, os vértices representam as estações de energia e as arestas representam as linhas de transmissão de energia.


\end{itemize}



\end{frame}


\begin{frame}{Passo 1 - Apresentação da Atividade}


\begin{itemize}
   
\item <1-> Considere o seguinte exemplo com 6 estações de energia $\{e_1,e_2,e_3,e_4,e_5, e_6\}$ e 7 linhas de transmissão $\{ e_1 \leftrightarrow e_2 ,e_1 \leftrightarrow e_5, e_2 \leftrightarrow e_3, e_3 \leftrightarrow e_4, e_4 \leftrightarrow e_5, e_5 \leftrightarrow e_6, e_6 \leftrightarrow e_2\}$:


\begin{tikzpicture}[main/.style = {draw, circle}] 

\node[main] (1) at (0,0) {$e_1$};
\node[main] (2) at (-1,-1) {$e_2$};
\node[main] (5) at (1,-1) {$e_5$};
\node[main] (3) at (-1,-2.5) {$e_3$};
\node[main] (4) at (1,-2.5) {$e_4$};
\node[main] (6) at (0,-3.5) {$e_6$};


\draw (1) -- (2);
\draw (1) -- (5);
\draw (2) -- (3);
\draw (3) -- (4);
\draw (4) -- (5);
\draw (5) -- (6);
\draw (6) -- (2);








\end{tikzpicture} 


\item <2-> Perceba que este sistema está normal, podemos encontrar uma rota entre qualquer par de estações.


\end{itemize}



\end{frame}


\begin{frame}{Passo 1 - Apresentação da Atividade}


\begin{itemize}
   
\item <1-> Considere o seguinte exemplo com 6 estações de energia $\{e_1,e_2,e_3,e_4,e_5, e_6\}$ e 7 linhas de transmissão $\{ e_1 \leftrightarrow e_2 ,e_1 \leftrightarrow e_5, e_2 \leftrightarrow e_3, e_3 \leftrightarrow e_4, e_4 \leftrightarrow e_5\}$:


\begin{tikzpicture}[main/.style = {draw, circle}] 

\node[main] (1) at (0,0) {$e_1$};
\node[main] (2) at (-1,-1) {$e_2$};
\node[main] (5) at (1,-1) {$e_5$};
\node[main] (3) at (-1,-2.5) {$e_3$};
\node[main] (4) at (1,-2.5) {$e_4$};
\node[main] (6) at (0,-3.5) {$e_6$};


\draw (1) -- (2);
\draw (1) -- (5);
\draw (2) -- (3);
\draw (3) -- (4);
\draw (4) -- (5);








\end{tikzpicture} 


\item <2-> Perceba que este sistema está estado de falha. Por exemplo,  não conseguimos encontrar uma rota interligando as estações $e_1$ e $e_6$.


\end{itemize}



\end{frame}


\begin{frame}{Passo 1 - Execução da atividade}


\begin{itemize}
   
\item <1-> A sala de aula pode ser dividida em equipes. 

\item <2-> No modo batalha, teremos o duelo entre duas equipes. 

\item <3-> A equipe 1 deve desenhar um diagrama de uma rede de transmissão com até 10 estações que deverá ser mantido em segredo.

\item <4-> A equipe2 deve descobrir se a rede de transmissão desenhada pela equipe1 está normal ou em estado de falha. 

\item <5-> A equipe2 pode realizar perguntas do tipo:
"Quais são as estações interligadas com a estação i?".

\item <6-> A equipe2 deve descobrir a condição da rede de transmissão realizando o número mínimo de perguntas.


\end{itemize}



\end{frame}


\begin{frame}{Passo 3 - Discussão e Avaliação}

\begin{itemize}

\item<1-> Os alunos são incentivados a escrever sobre as dificuldades para realizar o procedimento.

\item<2-> Os seguintes questionamentos podem ser feitos:

\begin{itemize}
    \item Quando podemos parar de fazer perguntas?
    \item Existe uma quantidade mínima de perguntas?
    \item A quantidade de perguntas está relacionada com a quantidade de linhas de transmissões ou com a quantidade de estações de energia?
\end{itemize}


\end{itemize}


\end{frame}





\end{document}