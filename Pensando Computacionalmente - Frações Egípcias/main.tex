\documentclass{beamer}
\mode<presentation>


\usepackage[brazil]{babel}
\usepackage[utf8]{inputenc}
 


\usepackage{amsfonts}
\usepackage{amssymb}
\usepackage{amsmath}
\usepackage{algorithm}
\usepackage{algpseudocode}



\usepackage{ae}
\usepackage{graphicx,color}
\usepackage[all]{xy}
\usepackage{empheq}
\usepackage{fancybox}
\usepackage{textcomp}
\usepackage[all]{xy}
\usepackage{textpos}
\usepackage{multicol}
\usepackage{cancel}
\usepackage{listings}
\usepackage{xcolor}
\usepackage{enumerate}
\usepackage{minted}

\usepackage[style=verbose]{biblatex}
\addbibresource{bibtex.bib}

\usepackage{tikz}
\usetikzlibrary{shapes.geometric, arrows}


\tikzstyle{startstop} = [rectangle, rounded corners, minimum width=3cm, minimum 
\tikzstyle{io} = [trapezium, trapezium left angle=70, trapezium right angle=110, minimum width=3cm, minimum height=1cm, text centered, draw=black, fill=blue!30]
\tikzstyle{process} = [rectangle, minimum width=3cm, minimum height=1cm, text centered, draw=black, fill=orange!30]
\tikzstyle{decision} = [diamond, minimum width=3cm, minimum height=1cm, text centered, draw=black, fill=green!30]
\tikzstyle{arrow} = [thick,->,>=stealth]

\newcommand{\floor}[1]{$\lfloor$ #1 $\rfloor$}

\newcommand\Fontvi{\fontsize{9}{7.2}\selectfont}



\usetheme{Boadilla}

\newcommand{\PC}[1]{\ensuremath{\left(#1\right)}}


\newcommand*{\colorboxed}{}
\def\colorboxed#1#{%
  \colorboxedAux{#1}%
}
\newcommand*{\colorboxedAux}[3]{%
  % #1: optional argument for color model
  % #2: color specification
  % #3: formula
  \begingroup
    \colorlet{cb@saved}{.}%
    \color#1{#2}%
    \boxed{%
      \color{cb@saved}%
      #3%
    }%
  \endgroup
}



\title {Pensando Computacionalmente}

\author[Wladimir Araújo Tavares]{ Wladimir Araújo Tavares$^{1}$  }

\institute[UFC]{$^{1}$Universidade Federal do Ceará - Campus de Quixadá\\}
\date{}
\AtBeginSection[]
{
  \begin{frame}<beamer>{}
    \small
    \tableofcontents[currentsection,currentsubsection]
  \end{frame}
}
\begin{document}

\begin{frame}
	\titlepage
\end{frame}

%%%%%%%%%%%%%%%%%%%%%%%%%%%%%%%%%%%%%%%%%%%%%%%%%%%%%%%%%%%%%%%%%%%%



\begin{frame}{Frações Egípcias}

\begin{itemize}
\item \textbf{Objetivos:} Desenvolver o pensamento computacional.

\item \textbf{Público-alvo:}  Alunos a partir do primeiro ano do Ensino Médio.

\item \textbf{Conteúdo:} Repetição

\item \textbf{Tempo:} 50 minutos

\item \textbf{Recursos:} Papel, Caneta.

\end{itemize}
    
\end{frame}


\begin{frame}{Passo 1 - Apresentação da Atividade}

\begin{itemize}
   
\item <1-> Os egípcios antigos não escreviam frações com numeradores maiores que 1. 

\item Dessa maneira, as frações com numeradores diferentes de 1 precisavam ser escritas como a soma de frações com numeradores iguais a 1.

\item Vamos tentar realizar a divisão da maneira egípcia.

\end{itemize}

\end{frame}


\begin{frame}{Passo 1 - Dividindo 3 pães para 4 pessoas}


\end{frame}


\begin{frame}{Passo 1 - Dividindo 5 pães para 8 pessoas}


\end{frame}


\begin{frame}{Passo 1 - Dividindo 8 pães para 9 pessoas}


\end{frame}

\begin{frame}{Passo 1 - Dividindo 8 pães para 11 pessoas}


\end{frame}


\begin{frame}{Passo 2 - Execução da atividade}

Os alunos devem resolver os seguintes problemas da maneira egípcia:

\begin{itemize}

\item Dividir 1 pão para 5 pessoas
\item Dividir 2 pães para 5 pessoas
\item Dividir 3 pães para 5 pessoas.
\item Dividir 4 pães para 5 pessoas.

\end{itemize}


\end{frame}






\begin{frame}{Passo 3 - Discussão e Avaliação}

\begin{itemize}

\item<1-> Os alunos são incentivados a escrever sobre as dificuldades para realizar o procedimento.

\item<2-> Será que o procedimento deveria ser melhor descrito? Será que faltou uma descrição mais detalhada.

\item <3-> Os alunos são incentivados a escrever no formato de um fluxograma o processo descrito acima.

\item<4-> O professor pode avaliar a participação do alunos em todos os passos da atividade.

\end{itemize}


\end{frame}





\end{document}