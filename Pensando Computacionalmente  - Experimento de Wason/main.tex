\documentclass{beamer}
\mode<presentation>


\usepackage[brazil]{babel}
\usepackage[utf8]{inputenc}
 

\usepackage{amsfonts}
\usepackage{amssymb}
\usepackage{amsmath}
\usepackage{algorithm}
\usepackage{algpseudocode}



\usepackage{ae}
\usepackage{graphicx,color}
\usepackage[all]{xy}
\usepackage{empheq}
\usepackage{fancybox}
\usepackage{textcomp}
\usepackage[all]{xy}
\usepackage{textpos}
\usepackage{multicol}
\usepackage{cancel}
\usepackage{listings}
\usepackage{xcolor}
\usepackage{tikz}
\usepackage{enumerate}
\usepackage{minted}

\usepackage[style=verbose]{biblatex}
\addbibresource{bibtex.bib}

\usetikzlibrary{matrix}
\usetikzlibrary{arrows}
\usetikzlibrary{shapes,snakes}
\usetikzlibrary{calc}


\newcommand{\floor}[1]{$\lfloor$ #1 $\rfloor$}

\newcommand\Fontvi{\fontsize{9}{7.2}\selectfont}



\usetheme{Boadilla}

\newcommand{\PC}[1]{\ensuremath{\left(#1\right)}}


\newcommand*{\colorboxed}{}
\def\colorboxed#1#{%
  \colorboxedAux{#1}%
}
\newcommand*{\colorboxedAux}[3]{%
  % #1: optional argument for color model
  % #2: color specification
  % #3: formula
  \begingroup
    \colorlet{cb@saved}{.}%
    \color#1{#2}%
    \boxed{%
      \color{cb@saved}%
      #3%
    }%
  \endgroup
}



\title {Pensando Computacionalmente}

\author[Wladimir Araújo Tavares]{ Wladimir Araújo Tavares$^{1}$  }

\institute[UFC]{$^{1}$Universidade Federal do Ceará - Campus de Quixadá\\}
\date{}
\AtBeginSection[]
{
  \begin{frame}<beamer>{}
    \small
    \tableofcontents[currentsection,currentsubsection]
  \end{frame}
}
\begin{document}

\begin{frame}
	\titlepage
\end{frame}

%%%%%%%%%%%%%%%%%%%%%%%%%%%%%%%%%%%%%%%%%%%%%%%%%%%%%%%%%%%%%%%%%%%%



\begin{frame}{Pensamento Lógico}

\begin{itemize}
\item \textbf{Objetivos:} Desenvolver o pensamento lógico e abstrato.

\item \textbf{Público-alvo:}  Alunos a partir do primeiro ano do Ensino Médio.

\item \textbf{Conteúdo:} Regras de se P então Q.

\item \textbf{Tempo:} 30 minutos

\item \textbf{Recursos:} Quadro e pincel

\end{itemize}
    
\end{frame}


% Dois experimentos foram realizados para investigar a dificuldade de fazer a inferência contrapositiva a partir de sentenças condicionais da forma “se P então Q”. Essa inferência,
% que não-P segue de não-Q, requer a transformação da informação apresentada
% na sentença condicional. Sugere-se que a dificuldade se deve a um conjunto mental de
% esperando uma relação de verdade, correspondência ou correspondência entre sentenças e
% estados de coisas. A elicitação da inferência não foi facilitada pela tentativa de
% induzir dois tipos de terapia destinados a quebrar este conjunto. Argumenta-se que os sujeitos
% não deu evidência de ter adquirido as características do “pensamento operacional formal” de Piaget.


\begin{frame}{Passo 1 - Apresentação da Atividade}

\begin{itemize}
   
\item Peter Wason \footfullcite{wason1968reasoning} propõs dois experimentos para investigar a dificuldade de realizar inferências a partir de sentenças condicionais da forma "se P então Q".

\item A atividade consiste em apresentar os dois experimentos apresentados por Wason. 

\item Todos os alunos devem anotar suas respostas no papel.

\end{itemize}

\end{frame}


\begin{frame}{Experimento I}


\begin{itemize}
    \item <1->Quatro cartas são apresentadas com uma letra em um lado e um número no outro lado.

\begin{center}
\begin{tikzpicture}
\node[rectangle, draw] (problema) at (0,0){3};
\node[rectangle, draw] (candidato) at (2,0) {d};
\node[rectangle, draw] (teste) at (4,0) {4};
\node[rectangle, draw] (teste) at (6,0) {f};
\end{tikzpicture}
\end{center}

\item <2->A tarefa é selecionar somente aquelas cartas que precisam ser viradas, para determina se o seguinte regra é válida:

\begin{center}
Se existe um $d$ em um lado, então existe um $3$ no outro lado.
\end{center}


\end{itemize}


\end{frame}


% The task is to select those and only those cards that need to be turned over, to
% determine whether the following conditional holds:
% If there is a d on one side,
% then there is a 3 on the other side.

% são quatro pessoas. Podemos ver o que dois deles estão bebendo, mas não quantos anos eles
% está; e podemos ver a idade de dois deles, mas não o que eles estão bebendo: 


% Bob, drinking beer.
% Mary, a senior citizen, obviously over eighteen years old.28
% Computational Logic and Human Thinking
% John, drinking cola.
% Susan, a primary school child, obviously under eighteen years old.
% In contrast with the card version of the selection task, most people solve th

\begin{frame}{Experimento II}

\begin{itemize}
    \item <1->Quatro pessoas estão em um bar. Você pode ver o que duas delas estão bebendo, mas não quanto anos elas têm. Você pode ver a idade das outras duas, mas não pode ser o que elas estão bebendo.

\begin{itemize}
    \item João está bebendo cerveja.
    \item Maria, uma idosa, obviamente com mais de dezoito anos.
    \item João está bebendo coca-cola.
    \item Giovana, uma criança da escola primária, obviamente com menos de dezoito anos.
\end{itemize}

\item <2->A tarefa é selecionar somente aquelas pessoas que precisam ser abordadas (perguntar a idade ou o que está bebendo), para determina se o seguinte regra é válida:

\begin{center}
Se uma pessoa está bebendo bebida alcoólica em um bar, então essa pessoa tem pelo menos 18 anos de idade.
\end{center}


\end{itemize}


\end{frame}

% If a person is drinking alcohol in a bar,
% then the person is at least eighteen years old.

\begin{frame}{Passo 2 - Compilando os resultados}

\begin{itemize}

\item Os resultados dos dois experimentos devem ser compilados.

\item Você pode fazer esse experimento agora.

\end{itemize}


\end{frame}


\begin{frame}{Passo 3 - Apresentação da resposta Experimento I}

\begin{itemize}

\item <1->A maioria das pessoas reconhece que o cartão com a letra d deve ser virado para verificar se o número 3 está no outro lado.

\item <2->Poucas pessoas reconhecem que o cartão com o número 7 também deve ser virado. Se atrás do cartão de número 7 tem a letra d então a regra estaria inválida.

\item <3->A regra original é equivalente logicamente a sua contrapositiva:

\begin{center}
Se não existe um $3$ em um lado, então não existe um $d$ no outro lado.
\end{center}

\item Apenas 10\% das pessoas acertaram a resposta desse experiemento.

\end{itemize}


\end{frame}


\begin{frame}{Passo 4 - Apresentação da resposta Experimento II}

\begin{itemize}

\item Já no experimento II, a maioria das pessoas resolve o desafio corretamente:

\begin{itemize}
    \item Precisamos checar se João tem pelo menos 18 anos.
    \item Precisamos checar se  Giovana está bebendo uma bebida sem álcool.
\end{itemize}



\end{itemize}


\end{frame}

% Psicólogos cognitivos propuseram um número desconcertante de teorias para
% explicar por que as pessoas são muito melhores em resolver essas versões da seleção
% tarefa em comparação com outras variações formalmente equivalentes, como o cartão original
% versão. A mais citada dessas teorias, devido a Leda Cosmides (1985,
% 1989), é que os humanos desenvolveram um algoritmo especializado (ou procedimento) para
% detectar trapaceiros em contratos sociais.


\begin{frame}{Passo 5 - Discussão}

\begin{itemize}

\item<1-> Nesse passo, os alunos são convidados a criarem suas teorias sobre a diferença dos resultados dos dois experimentos.

\item<2-> Psicólogos cognitivos acreditam que os humanos desenvolveram um algoritmo especializado para detectar trapaças em contratos sociais.

\item<3-> Cosmides\footfullcite{cosmides1989logic} argumenta que os humanos desenvolveram outros algoritmos especializados para lidar com outros tipos de problemas, por exemplo:


Se você se envolver em uma atividade perigosa,
então você deve tomar as devidas precauções.


\end{itemize}


\end{frame}
% \section{Encontrando o maior}

% \begin{frame}{Tarefa}

% \begin{itemize}

% \item Dado um conjunto de valores, encontre o maior dos valores.

% \begin{center}
% \begin{tabular}{|c|c|c|c|c|}
% \hline
% 10    &  15 & 30 & 42 & 14 \\
% \hline
% \end{tabular}
% \end{center}

% \item Você pode dizer que esse problema é muito fácil e a resposta claramente é 42. 

% \item Como você resolveria esse problema com os olhos fechados e podendo fazer perguntas simples para uma outra pessoa.

% \end{itemize}


% \end{frame}



% \begin{frame}{Decomposição}

% \begin{itemize}

% \item Sabemos que a resposta é o elemento que é maior que todos os outros.

% \item Nesse problema, precisaremos guardar uma informação que chamaremos de candidato, utilizaremos o candidato para comparar com cada um dos valores do nosso conjunto.

% \item Possivelmente, esse candidato pode ser desbancado por um dos valores do nosso conjunto e terá que ser atualizado.


% \end{itemize}


% \end{frame}


% \begin{frame}{Decomposição}


% \begin{tikzpicture}
% \node[rectangle, draw] (problema) at (0,0){Encontrar o maior elemento do conjunto};
% \node[rectangle, draw] (candidato) at (0,-1) {Selecionar um 
% candidato inicial};
% \node[rectangle, draw] (teste) at (1,-3) {Testar se o elemento $i$ desbanca o candidato e atualizar o candidato}; 
% \draw[->]  (problema) -- (candidato);
% \draw[->]  (problema) -- (teste);

% \end{tikzpicture}



% \end{frame}


% \begin{frame}{Reconhecimento de Padrões}


% \begin{itemize}
%     \item O subproblema de testar se um elemento desbanca o candidato atual e atualiza o candidato atual é o mesmo subproblema para todos os elementos do conjunto.
%     \item O que pode mudar entre os testes para cada elemento é que o candidato pode ter sido atualizado.
% \end{itemize}


% \end{frame}


% \begin{frame}{Abstração}


% \begin{itemize}
%     \item Os valores podem ser armazenados em um vetor e podem ser acessados por um índice começado por 1 até o tamanho do conjunto.
%     \item Para realizar o teste e atualização, precisamos acessar o elemento a ser testado e do candidato atual.
% \end{itemize}


% \end{frame}

% \begin{frame}{Algoritmo em pseudocódigo}


% \begin{algorithm}[H]
% \caption{maior\_elemento(A)}
% \begin{algorithmic}
% \Require o tamanho do vetor A deve ser maior que 1.
% \Ensure Devolve o maior elemento do vetor

% \State $candidato \gets A[1]$
% \State $n \gets tamanho(A)$

% \For{$i \gets 1$ \textbf{até} $n$}

% \If{A[i] > candidato}
% \State $candidato \gets A[i]$
% \EndIf

% \EndFor

% \State \Return A[i]

% \end{algorithmic}
% \end{algorithm}

% \end{frame}

% \begin{frame}[fragile]{Algoritmo utilizando a linguagem Python}

% \begin{minted}{Python}[H]
% def maior_elemento(A):
% 	candidato = A[0]
% 	n = len(A)
% 	for i in range(n):
% 		if A[i] > candidato:
% 			candidato = A[i]
% 	return candidato
% if __name__ == "__main__":
% 	print( maior_elemento([4,10,42,15,30]))	

% \end{minted}

% \end{frame}



\end{document}